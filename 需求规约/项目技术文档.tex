\documentclass{article}

\usepackage{fancyvrb}

\newenvironment{markdown}%
    {\VerbatimEnvironment\begin{VerbatimOut}{tmp.markdown}}%
    {\end{VerbatimOut}%
        \immediate\write18{pandoc tmp.markdown -t latex -o tmp.tex}%
        \input{tmp.tex}}

\begin{document}

\begin{markdown}

# 项目技术文档

## 开发方法

主要采用基于 UML 的面向对象方法进行开发。

## 项目技术架构

Browser/Server 架构(Web 版本)

Client/Server 架构(App 客户端版本)

## 项目技术层次

### 视图层

* 基于 `Angular`、 `Vue`、和 `React` 的前端 `Web` 页面

* 基于 `Cocoa` 的 `iOS` 客户端

* 基于 `Android SDK` 的 `Android` 客户端

* 主要基于 `JavaScript` 的微信小程序(WeChat Mini App)

本层直接和业务层双向相关。

### 业务层

* 基于 `Spring Cloud` 的后端服务器

* 基于 `SpringBoot JPA` 的数据管理

* 基于 `Servlet` 的 API 接口实现

* 基于 `Jenkins` 的持续集成服务

本层直接和视图层和数据层相关。

### 数据层

* 基于 `MySQL` 和 `MongoDB` 的数据持久化存储

本层直接和业务层相关。

## 项目技术栈

### Web 前端开发

* `JavaScript` / `TypeScript`

* `Vue` / `React` / `Angular`

* `npm`

### iOS App 开发

* `Swift` / `Objective-C`

* `CocoaPods`

### Android App 开发

* `Java` / `Kotlin`

### 服务器后端开发

* `Spring Series`

* `Java` / `Kotlin`

* `Maven`

* `Servlet`

### 数据库管理系统

* `MySQL`

* `SpringBoot JPA`

\end{markdown}

\end{document}