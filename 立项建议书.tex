\documentclass{article}

\usepackage{fancyvrb}

\newenvironment{markdown}%
    {\VerbatimEnvironment\begin{VerbatimOut}{tmp.markdown}}%
    {\end{VerbatimOut}%
        \immediate\write18{pandoc tmp.markdown -t latex -o tmp.tex}%
        \input{tmp.tex}}

\begin{document}

\begin{markdown}
# 优邻购立项建议书

## 项目组 优邻购项目组


| 学号 | 姓名 | 手机 | 电子邮箱 |
|---|---|---|---|
| 517030910237 | 高鹏成 | 15221275159 | 1271569621@qq.com |
| 517030910168 | 于喜千 | 17465890308 | akaza_akari@sjtu.edu.cn |
| 517030910210 | 何思泽 | 18701961157 | 1217286706@qq.com |
| 517030910033 | 高鸿博 | 18047709309 | 2037210877@qq.com |

## 一、项目必要性

在O2O零售行业快速崛起的背景下,茅台酒厂目前的销售方式仍然是授予线下经销商代理资格。优邻购是将之前代理商的零售店统一起来,构建一个O2O零售平台,并要求代理商通过优邻购平台进行销售。优邻购让公司从只管理代理商卖酒到代理商基于优邻购平台运营,拓展了公司的业务,也直接能购提高公司的利润。让代理商基于公司的平台运营也能减少公司产品到消费者的中间环节,降低成本,提高公司的竞争力。
目前,公司的代理商众多,让他们进驻优邻购可以迅速占据大部分O2O零售的市场份额,前景广阔。

## 二、项目目标和特性

### 2.1 主要目标

优邻购项目的目标是建立一个在线商品零售平台,代理商上架商品,消费者购买商品,由公司进行管理。借此拓展公司业务。

### 2.2 定位

优邻购的定位是作为一个代理商和消费者的线上零售平台,为代理商和消费者提供各种零售服务。

### 2.3 功能性需求——优先级

1. 管理员管理经销商信息—高

2. 管理员管理店铺信息—高

3. 经销商修改部分账户信息—高

4. 经销商管理自己的店铺信息—高

5. 经销商管理自己的店铺的商品信息—高

6. 经销商将商品上架到优邻购平台—高

7. 经销商收到订单通知—中

8. 经销商选择配送方式—中

9. 消费者注册管理账户—高

10. 消费者查看零售店列表—高

11. 消费者查看零售店列表优先级排序—中

12. 消费者浏览商品信息并将商品加入购物车或直接购买—高

13. 消费者通过购物车下单—高

14. 消费者查看订单状态—高

15. 消费者对店铺进行评价—中

16. 为消费者推荐商品和店铺—低

17. 在地图上显示配送距离—低

18. 按类别查看商品—中

19. 商品标签管理—低

20. 经销商查看销售情况—中

### 2.4 非功能性需求

1. 易用性 <br> 一个普通大学生可以在平均4分钟或最多6分钟内理解用户系统使用方法,能够注册并完成一次下单。一个普通大学生可以在平均6分钟或最多8分钟内理解经销商系统使用方法,能够在优邻购上架一件商品🍎。一个普通大学生可以在平均8分钟或最多10分钟内理解管理员系统使用方法,并能修改一个经销商的信息。
2. 可靠性 <br> 系统一年平均正常运行时间至少达到99.5%。
3. 性能 <br> 系统支持1000个并发用户,可以储存并展示3000种商品,1000家店铺,用户操作的响应时间不大于1.5秒—低
4. 可支持性 <br> 微服务便于维护。

## 三、项目技术方案
![Image 1](https://raw.githubusercontent.com/yulingo/paper-work/master/需求规约/Assets/技术架构图示-1.png)
![Image 2](https://raw.githubusercontent.com/yulingo/paper-work/master/需求规约/Assets/技术架构图示-2.png)
## 开发方法

主要采用基于 UML 的面向对象方法进行开发。

## 项目技术架构

Browser/Server 架构(Web 版本)

Client/Server 架构(App 客户端版本)

## 项目技术层次

### 视图层

* 基于 `Angular`、 `Vue`、和 `React` 的前端 `Web` 页面

* 基于 `Cocoa` 的 `iOS` 客户端

* 基于 `Android SDK` 的 `Android` 客户端

* 主要基于 `JavaScript` 的微信小程序(WeChat Mini App)

本层直接和业务层双向相关。

### 业务层

* 基于 `Spring Cloud` 的后端服务器

* 基于 `SpringBoot JPA` 的数据管理

* 基于 `Servlet` 的 API 接口实现

* 基于 `Jenkins` 的持续集成服务

本层直接和视图层和数据层相关。

### 数据层

* 基于 `MySQL` 和 `MongoDB` 的数据持久化存储

本层直接和业务层相关。

## 项目技术栈

### Web 前端开发

* `JavaScript` / `TypeScript`

* `Vue` / `React` / `Angular`

* `npm`

### iOS App 开发

* `Swift` / `Objective-C`

* `CocoaPods`

### Android App 开发

* `Java` / `Kotlin`

### 服务器后端开发

* `Spring Series`

* `Java` / `Kotlin`

* `Maven`

* `Servlet`

### 数据库管理系统

* `MySQL`

* `SpringBoot JPA`


## 四、项目风险分析和里程碑计划

### 4.1 项目风险:技术风险,进度风险,架构风险

1. 技术风险 <br> 对 SpringCloud 框架使用不够熟练,需要深入学习,对OAuth2.0 不够了解。
2. 进度风险 <br> 开发时间紧,因此采用四个冲刺(Sprint1、Sprint2、Sprint3、Sprint4)来增量式实现功能。
3. 架构风险 <br> 对微服务架构不了解。

### 4.2 迭代计划

根据以上阐释的风险,我们将项目实现的过程主要分成以下五个迭代:

| 迭代 | 任务描述 | 成果 |
|---|---|---|
| 项目启动 <br> 6月14日~6月30日| 学习SpringCloud,OAuth,微服务架构;开发界面原型;项目立项。 | 项目计划;界面原型 |
| Sprint 1 <br> 7月1日~7月14日 | 架构分析与设计;架构实现与搭建;R1的设计与实现; | 完成系统版本1(R1)的开发 |
| Sprint 2 <br> 7月15日~7月28日 | 在R1的基础上进行R2的设计与实现;系统测试,进行缺陷修复与改进。 | 完成系统版本2(R2)的开发 |
| Sprint 3 <br> 7月29日~8月2日 | 在R2的基础上实现进阶需求;系统测试,进行缺陷修复与改进。 | 完成系统版本3(R3)的开发 |
| Sprint 4 <br> 8月3日~9月8日 | 在R3的基础上进行进阶需求实现;系统测试,进行缺陷修复与改进。验收准备。 | 完成系统版本4(R4)的发布;完成演示视频、PPT等验收所需文件 |

系统版本1(R1)必须实现的功能:

* 管理员管理经销商信息
* 管理员管理店铺信息
* 经销商修改部分账户信息
* 消费者注册管理账户

系统版本2(R2)必须实现的功能:

* 经销商管理自己的店铺信息
* 经销商管理自己的店铺的商品信息
* 经销商将商品上架到优邻购平台
* 消费者查看零售店列表
* 消费者浏览商品信息并将商品加入购物车或直接购买
* 消费者通过购物车下单
* 消费者查看订单状态

系统版本3(R3)必须实现的功能:

* 经销商收到订单通知
* 经销商选择配送方式
* 消费者查看零售店列表优先级排序
* 消费者对店铺进行评价

系统版本4(R4)必须实现的功能:

* 为消费者推荐商品和店铺
* 在地图上显示配送距离
* 按类别查看商品
* 商品标签管理
* 经销商查看销售情况
* API权限管理
* 服务监控
* 日志管理

## 五、项目预期成果

* 《项目计划》
* 《迭代计划》(每个迭代开始前编写迭代计划)
* 《迭代评估报告》(每个迭代结束后编写迭代评估报告)
* 《SRS文档》和用例模型(.oom)
* 《软件架构文档》和分析设计模型(.oom)
* 《测试用例》和《测试报告》
* 《项目总结报告》
* 源代码和可执行代码
* 演示视频文件(包括安装、运行、功能等)
* 演示PPT


\end{markdown}

\end{document}